The CBF shall perform channelisation such that the 53 dB attenuation
bandwidth is $\le 2\times$ (twice) the pass band width.
